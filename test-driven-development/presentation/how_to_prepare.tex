\documentclass{article}

\usepackage{amsmath}
\usepackage{tikz}
\usepackage{listings}
\usepackage[
	colorlinks=true,
	linkcolor=blue
]{hyperref}

% Overlays for TIKZ pictures
\tikzset{
  invisible/.style={opacity=0},
  visible on/.style={alt={#1{}{invisible}}},
  alt/.code args={<#1>#2#3}{%
    \alt<#1>{\pgfkeysalso{#2}}{\pgfkeysalso{#3}} % \pgfkeysalso doesn't change the path
  },
}

\title{Preparation for workshop "Clean architecture"}
\author{Guido Plessmann, Pierre-Francois Duc, Hendrik Huyskens}

\begin{document}

\maketitle

\section{Short infos}
\begin{itemize}
\item The workshop addresses (python-) developers with at least \textbf{average programming skills}
\item Please bring your \textbf{laptop} to the workshop
\item As there might be no internet - please follow installation instructions (below) \textbf{before} the workshop
\item Feel free to contact me (Hendrik) if you need help!
\end{itemize}

\section{Installation}

\begin{itemize}
\item Install conda (\href{https://docs.conda.io/en/latest/miniconda.html}{miniconda} or \href{https://www.anaconda.com/distribution/}{anaconda})
\item Clone/download \href{https://github.com/rl-institut/tdd_workshop}{workshop repository}:
\begin{lstlisting}
git clone https://github.com/rl-institut/tdd_workshop.git
\end{lstlisting}
\item (optional) rename conda environment in environment.yml file
\lstinputlisting{../environment.yml}
\item 
\href{https://docs.conda.io/projects/conda/en/latest/user-guide/tasks/manage-environments.html\#creating-an-environment-from-an-environment-yml-file}{Install conda environment from file} (open miniconda/anaconda prompt terminal):
\begin{lstlisting}
conda env create -f environment.yml
\end{lstlisting}
\end{itemize}


\end{document}
