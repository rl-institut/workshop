\documentclass{beamer}


\usetheme{default}      % or try Darmstadt, Madrid, Warsaw, ...
\usecolortheme{default} % or try albatross, beaver, crane, ...
\usefonttheme{default}  % or try serif, structurebold, ...
\setbeamertemplate{navigation symbols}{}
\setbeamertemplate{caption}[numbered]
\setbeamertemplate{itemize items}{\raisebox{-0.em}{\scalebox{1}{\textbullet}}}


\usepackage[english]{babel}
\usepackage[utf8x]{inputenc}


%%%% adjustments to approximate RLI-template

% Colors
\definecolor{rliblue}{HTML}{002E4F}
\setbeamercolor{title}{fg=rliblue}
\setbeamercolor{frametitle}{fg=rliblue}
\setbeamercolor{normal text}{fg=rliblue}
\setbeamercolor{alerted text}{fg=rliblue}
\setbeamercolor{section in head/foot}{bg=rliblue}
\setbeamercolor{author in head/foot}{bg=rliblue}
\setbeamercolor{date in head/foot}{fg=rliblue}

% Itemize environment
\setbeamertemplate{itemize item}{\color{rliblue}$\blacktriangleright$}
\setbeamertemplate{itemize subitem}{\color{rliblue}$\blacktriangleright$}
\setbeamercolor{enumerate item}{fg=rliblue}
\setbeamercolor{enumerate subitem}{fg=rliblue}

% Fonts
\setbeamerfont{author}{size={\fontsize{10}{10}}}

\pgfdeclareimage[width=\paperwidth]{mybackground}{pics/background.pdf}
\setbeamertemplate{title page}{
        \begin{picture}(0,0)
            \put(-29,-157){%
                \pgfuseimage{mybackground}
            }

            \put(0,-110.7){%
                \begin{minipage}[b][45mm][t]{.5\textwidth}
                    \usebeamerfont{title}{\inserttitle\par\vspace{1.cm}}
                    \usebeamerfont{author}{\insertauthor\par}
                    \usebeamerfont{institute}{\insertinstitute\par}
                    \usebeamerfont{date}{\insertdate\par}
                \end{minipage}
            }
        \end{picture}
    }
    

\usepackage{tikz}
\addtobeamertemplate{headline}{}{%
\begin{tikzpicture}[remember picture,overlay]
\node at([shift={(.40\paperwidth,-.80)}]current page.north) {\includegraphics[scale=0.1]{pics/RLI_logo.pdf}};
\end{tikzpicture}}

\usepackage{multicol}

\setbeamertemplate{footline}[frame number]
\usepackage{amsmath}
\usepackage{tikz}

% Overlays for TIKZ pictures
\tikzset{
  invisible/.style={opacity=0},
  visible on/.style={alt={#1{}{invisible}}},
  alt/.code args={<#1>#2#3}{%
    \alt<#1>{\pgfkeysalso{#2}}{\pgfkeysalso{#3}} % \pgfkeysalso doesn't change the path
  },
}

\title[Yo]{\textbf{SimRunde Workshop} \\ Python testing}
\author{Pierre-Francois Duc\\
}
\institute{Reiner Lemoine Institut}
\date{\today}

\begin{document}

\begin{frame}
  \titlepage
\end{frame}

\begin{frame}{Agenda}
 \begin{enumerate}
  \item 2022-02-02: Unit test and test driven development
  \item 2022-02-09: Continuous integration and automated tests on github
 \end{enumerate}

\end{frame}

\begin{frame}{Why testing?}
 \begin{enumerate}
  \item To not ask yourself why you did not test when your code does not work 2 weeks before a project's end
  \item To feel confident that your code is doing what you are expecting from it
 \end{enumerate}

 Coding without testing is like climbing without rope: it is fun and you feel free, but it is risky and you can hurt yourself badly :(
 
\end{frame}

\begin{frame}{Unit Tests: Idea}
Goal:
\begin{itemize}
\item Isolate each part of the program and show that the individual parts are correct
\item Allows refactoring of code at a later date, and makes sure the module is still working correctly
\end{itemize}
How:
\begin{itemize}
\item Many small independent and isolated tests
\item Automated tests run by developer
\item Used in TDD
\end{itemize}
\end{frame}

\begin{frame}{Unit Tests: Rules}
Unit tests should be:
\begin{description}

\item[small] e.g. concentrate on single criteria, so that failing of a test directly points to underlying error \\
Rule of thumb: Consider if the test is a logical AND between conditions or a logical OR . In the former case go for multiple assertions, in the latter create multiple test functions.
\item[fast] in order to run whole test set after each change/save to the code
\item[idempotent] e.g. should be order-independent and not rely on each other
\item[isolated] e.g. should be independent of environment or external APIs
\end{description}
\end{frame}


\begin{frame}{Other Tests}
\begin{description}
\item[Integration Test] \hfill\\
Occurs after unit testing; combines multiple unit tests to a group and tests group outcome.\\
Example:\\
Login at GUI $\rightarrow$ API call $\rightarrow$ DB check $\rightarrow$ login
\item[Regression Test] \hfill \\
Re-running functional and non-functional tests to ensure that previously developed and tested software still performs after a change.
\item[Smoke Test] \hfill\\
Smoke tests are a subset of test cases that cover the most important functionality of a component or system, used to aid assessment of whether main functions of the software appear to work correctly.
\item[...]
\end{description}
\end{frame}


\begin{frame}{Now some demonstration}
We have the requirements to write a function which can add up numbers. Let's do it following the TTD philosophy, step by step.
\end{frame}


\begin{frame}{Follow-up: next workshop}
Everyone is welcome to join

\vspace{15pt}

It will happen on 2022-02-09 1230-1400 and we will see what type of tests one could implement on an existing repository and how to setup automated tests with github actions.
 
\end{frame}



\end{document}

